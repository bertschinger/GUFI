% This file is part of GUFI, which is part of MarFS, which is released
% under the BSD license.
%
%
% Copyright (c) 2017, Los Alamos National Security (LANS), LLC
% All rights reserved.
%
% Redistribution and use in source and binary forms, with or without modification,
% are permitted provided that the following conditions are met:
%
% 1. Redistributions of source code must retain the above copyright notice, this
% list of conditions and the following disclaimer.
%
% 2. Redistributions in binary form must reproduce the above copyright notice,
% this list of conditions and the following disclaimer in the documentation and/or
% other materials provided with the distribution.
%
% 3. Neither the name of the copyright holder nor the names of its contributors
% may be used to endorse or promote products derived from this software without
% specific prior written permission.
%
% THIS SOFTWARE IS PROVIDED BY THE COPYRIGHT HOLDERS AND CONTRIBUTORS "AS IS" AND
% ANY EXPRESS OR IMPLIED WARRANTIES, INCLUDING, BUT NOT LIMITED TO, THE IMPLIED
% WARRANTIES OF MERCHANTABILITY AND FITNESS FOR A PARTICULAR PURPOSE ARE DISCLAIMED.
% IN NO EVENT SHALL THE COPYRIGHT HOLDER OR CONTRIBUTORS BE LIABLE FOR ANY DIRECT,
% INDIRECT, INCIDENTAL, SPECIAL, EXEMPLARY, OR CONSEQUENTIAL DAMAGES (INCLUDING,
% BUT NOT LIMITED TO, PROCUREMENT OF SUBSTITUTE GOODS OR SERVICES; LOSS OF USE,
% DATA, OR PROFITS; OR BUSINESS INTERRUPTION) HOWEVER CAUSED AND ON ANY THEORY OF
% LIABILITY, WHETHER IN CONTRACT, STRICT LIABILITY, OR TORT (INCLUDING NEGLIGENCE
% OR OTHERWISE) ARISING IN ANY WAY OUT OF THE USE OF THIS SOFTWARE, EVEN IF
% ADVISED OF THE POSSIBILITY OF SUCH DAMAGE.
%
%
% From Los Alamos National Security, LLC:
% LA-CC-15-039
%
% Copyright (c) 2017, Los Alamos National Security, LLC All rights reserved.
% Copyright 2017. Los Alamos National Security, LLC. This software was produced
% under U.S. Government contract DE-AC52-06NA25396 for Los Alamos National
% Laboratory (LANL), which is operated by Los Alamos National Security, LLC for
% the U.S. Department of Energy. The U.S. Government has rights to use,
% reproduce, and distribute this software.  NEITHER THE GOVERNMENT NOR LOS
% ALAMOS NATIONAL SECURITY, LLC MAKES ANY WARRANTY, EXPRESS OR IMPLIED, OR
% ASSUMES ANY LIABILITY FOR THE USE OF THIS SOFTWARE.  If software is
% modified to produce derivative works, such modified software should be
% clearly marked, so as not to confuse it with the version available from
% LANL.
%
% THIS SOFTWARE IS PROVIDED BY LOS ALAMOS NATIONAL SECURITY, LLC AND CONTRIBUTORS
% "AS IS" AND ANY EXPRESS OR IMPLIED WARRANTIES, INCLUDING, BUT NOT LIMITED TO,
% THE IMPLIED WARRANTIES OF MERCHANTABILITY AND FITNESS FOR A PARTICULAR PURPOSE
% ARE DISCLAIMED. IN NO EVENT SHALL LOS ALAMOS NATIONAL SECURITY, LLC OR
% CONTRIBUTORS BE LIABLE FOR ANY DIRECT, INDIRECT, INCIDENTAL, SPECIAL,
% EXEMPLARY, OR CONSEQUENTIAL DAMAGES (INCLUDING, BUT NOT LIMITED TO, PROCUREMENT
% OF SUBSTITUTE GOODS OR SERVICES; LOSS OF USE, DATA, OR PROFITS; OR BUSINESS
% INTERRUPTION) HOWEVER CAUSED AND ON ANY THEORY OF LIABILITY, WHETHER IN
% CONTRACT, STRICT LIABILITY, OR TORT (INCLUDING NEGLIGENCE OR OTHERWISE) ARISING
% IN ANY WAY OUT OF THE USE OF THIS SOFTWARE, EVEN IF ADVISED OF THE POSSIBILITY
% OF SUCH DAMAGE.



\section{Indexing}
The first step to using GUFI is indexing a source filesystem.

An index created by GUFI retains the shape of the source filesystem:
directories that exist in the source filesystem also exist in the
index. If an administrator created the index, the directories will
also have the same access permissions, uid, and gid.

Indexes will not contain any of the files that the source filesystem
contained. Instead, all metadata extracted from a single directory of
the source filesystem will be placed into a single database file,
called db.db, in the cooresponding directory in the index. Each
database file will be created with a fixed schema that includes the
tables listed in Sections \ref{sec:schema} and
\ref{sec:xattr_schema}. Additional database files may be created if
extended attributes are extracted (see Section~\ref{sec:xattrs}).

Index processing (creation) occurs on a per-directory basis, and thus
is highly parallelizable.

\subsection{Directly Indexing a Filesystem}
% This file is part of GUFI, which is part of MarFS, which is released
% under the BSD license.
%
%
% Copyright (c) 2017, Los Alamos National Security (LANS), LLC
% All rights reserved.
%
% Redistribution and use in source and binary forms, with or without modification,
% are permitted provided that the following conditions are met:
%
% 1. Redistributions of source code must retain the above copyright notice, this
% list of conditions and the following disclaimer.
%
% 2. Redistributions in binary form must reproduce the above copyright notice,
% this list of conditions and the following disclaimer in the documentation and/or
% other materials provided with the distribution.
%
% 3. Neither the name of the copyright holder nor the names of its contributors
% may be used to endorse or promote products derived from this software without
% specific prior written permission.
%
% THIS SOFTWARE IS PROVIDED BY THE COPYRIGHT HOLDERS AND CONTRIBUTORS "AS IS" AND
% ANY EXPRESS OR IMPLIED WARRANTIES, INCLUDING, BUT NOT LIMITED TO, THE IMPLIED
% WARRANTIES OF MERCHANTABILITY AND FITNESS FOR A PARTICULAR PURPOSE ARE DISCLAIMED.
% IN NO EVENT SHALL THE COPYRIGHT HOLDER OR CONTRIBUTORS BE LIABLE FOR ANY DIRECT,
% INDIRECT, INCIDENTAL, SPECIAL, EXEMPLARY, OR CONSEQUENTIAL DAMAGES (INCLUDING,
% BUT NOT LIMITED TO, PROCUREMENT OF SUBSTITUTE GOODS OR SERVICES; LOSS OF USE,
% DATA, OR PROFITS; OR BUSINESS INTERRUPTION) HOWEVER CAUSED AND ON ANY THEORY OF
% LIABILITY, WHETHER IN CONTRACT, STRICT LIABILITY, OR TORT (INCLUDING NEGLIGENCE
% OR OTHERWISE) ARISING IN ANY WAY OUT OF THE USE OF THIS SOFTWARE, EVEN IF
% ADVISED OF THE POSSIBILITY OF SUCH DAMAGE.
%
%
% From Los Alamos National Security, LLC:
% LA-CC-15-039
%
% Copyright (c) 2017, Los Alamos National Security, LLC All rights reserved.
% Copyright 2017. Los Alamos National Security, LLC. This software was produced
% under U.S. Government contract DE-AC52-06NA25396 for Los Alamos National
% Laboratory (LANL), which is operated by Los Alamos National Security, LLC for
% the U.S. Department of Energy. The U.S. Government has rights to use,
% reproduce, and distribute this software.  NEITHER THE GOVERNMENT NOR LOS
% ALAMOS NATIONAL SECURITY, LLC MAKES ANY WARRANTY, EXPRESS OR IMPLIED, OR
% ASSUMES ANY LIABILITY FOR THE USE OF THIS SOFTWARE.  If software is
% modified to produce derivative works, such modified software should be
% clearly marked, so as not to confuse it with the version available from
% LANL.
%
% THIS SOFTWARE IS PROVIDED BY LOS ALAMOS NATIONAL SECURITY, LLC AND CONTRIBUTORS
% "AS IS" AND ANY EXPRESS OR IMPLIED WARRANTIES, INCLUDING, BUT NOT LIMITED TO,
% THE IMPLIED WARRANTIES OF MERCHANTABILITY AND FITNESS FOR A PARTICULAR PURPOSE
% ARE DISCLAIMED. IN NO EVENT SHALL LOS ALAMOS NATIONAL SECURITY, LLC OR
% CONTRIBUTORS BE LIABLE FOR ANY DIRECT, INDIRECT, INCIDENTAL, SPECIAL,
% EXEMPLARY, OR CONSEQUENTIAL DAMAGES (INCLUDING, BUT NOT LIMITED TO, PROCUREMENT
% OF SUBSTITUTE GOODS OR SERVICES; LOSS OF USE, DATA, OR PROFITS; OR BUSINESS
% INTERRUPTION) HOWEVER CAUSED AND ON ANY THEORY OF LIABILITY, WHETHER IN
% CONTRACT, STRICT LIABILITY, OR TORT (INCLUDING NEGLIGENCE OR OTHERWISE) ARISING
% IN ANY WAY OUT OF THE USE OF THIS SOFTWARE, EVEN IF ADVISED OF THE POSSIBILITY
% OF SUCH DAMAGE.



\subsubsection{\gufidirindex}
\gufidirindex is used to directly create an index based off of the
contents of a provided directory.

\begin{table} [h!]
  \centering
  \begin{tabular}{l|l}
    Flag & Functionality \\
    \hline
    -h & help manual \\
    \hline
    -H & Show assigned input values \\
    \hline
    -v & version \\
    \hline
    -n \textless num\_threads\textgreater & define number of threads to use \\
    \hline
    -x & index/query xattrs \\
    \hline
    -y \textless min level\textgreater & minimum level to go down \\
    \hline
    -z \textless max level\textgreater & maximum level to go down \\
    \hline
    -k \textless filename\textgreater & file containing directory names to skip \\
    \hline
    -M \textless bytes\textgreater & target memory footprint \\
    \hline
    -s \textless path\textgreater & File name prefix for swap files \\
    \hline
    -C \textless count\textgreater & Number of subdirectories allowed to be \\
                                   & enqueued for parallel processing. Any \\
                                   & remainders will be processed in-situ \\
    \hline
    -e & compress work items \\
    \hline
    -q \textless basename\textgreater & Basename of file to keep track of during indexing \\
    \hline
  \end{tabular}
  \caption{\label{fig:Flags_for_dir2index} \gufidirindex Flags and Arguments}
\end{table}

\begin{figure} [h!]
  \centering
  \includegraphics[width=1.0\textwidth]{images/gufi_dir2index.png}
  \caption{\label{fig:gufi_dir2index} \gufidirindex workflow}
\end{figure}

\paragraph{Usage} ~\\\\
\gufidirindex \texttt{[flags] input\_dir... /path/to/index} \\\\
The index of \texttt{input\_dir\textsubscript{i}} will be placed in \texttt{output\_dir/\$(basename input\_dir\textsubscript{i})}.

\subsection{Indirectly Indexing a Filesystem}
\input{sections/tracefile}
% This file is part of GUFI, which is part of MarFS, which is released
% under the BSD license.
%
%
% Copyright (c) 2017, Los Alamos National Security (LANS), LLC
% All rights reserved.
%
% Redistribution and use in source and binary forms, with or without modification,
% are permitted provided that the following conditions are met:
%
% 1. Redistributions of source code must retain the above copyright notice, this
% list of conditions and the following disclaimer.
%
% 2. Redistributions in binary form must reproduce the above copyright notice,
% this list of conditions and the following disclaimer in the documentation and/or
% other materials provided with the distribution.
%
% 3. Neither the name of the copyright holder nor the names of its contributors
% may be used to endorse or promote products derived from this software without
% specific prior written permission.
%
% THIS SOFTWARE IS PROVIDED BY THE COPYRIGHT HOLDERS AND CONTRIBUTORS "AS IS" AND
% ANY EXPRESS OR IMPLIED WARRANTIES, INCLUDING, BUT NOT LIMITED TO, THE IMPLIED
% WARRANTIES OF MERCHANTABILITY AND FITNESS FOR A PARTICULAR PURPOSE ARE DISCLAIMED.
% IN NO EVENT SHALL THE COPYRIGHT HOLDER OR CONTRIBUTORS BE LIABLE FOR ANY DIRECT,
% INDIRECT, INCIDENTAL, SPECIAL, EXEMPLARY, OR CONSEQUENTIAL DAMAGES (INCLUDING,
% BUT NOT LIMITED TO, PROCUREMENT OF SUBSTITUTE GOODS OR SERVICES; LOSS OF USE,
% DATA, OR PROFITS; OR BUSINESS INTERRUPTION) HOWEVER CAUSED AND ON ANY THEORY OF
% LIABILITY, WHETHER IN CONTRACT, STRICT LIABILITY, OR TORT (INCLUDING NEGLIGENCE
% OR OTHERWISE) ARISING IN ANY WAY OUT OF THE USE OF THIS SOFTWARE, EVEN IF
% ADVISED OF THE POSSIBILITY OF SUCH DAMAGE.
%
%
% From Los Alamos National Security, LLC:
% LA-CC-15-039
%
% Copyright (c) 2017, Los Alamos National Security, LLC All rights reserved.
% Copyright 2017. Los Alamos National Security, LLC. This software was produced
% under U.S. Government contract DE-AC52-06NA25396 for Los Alamos National
% Laboratory (LANL), which is operated by Los Alamos National Security, LLC for
% the U.S. Department of Energy. The U.S. Government has rights to use,
% reproduce, and distribute this software.  NEITHER THE GOVERNMENT NOR LOS
% ALAMOS NATIONAL SECURITY, LLC MAKES ANY WARRANTY, EXPRESS OR IMPLIED, OR
% ASSUMES ANY LIABILITY FOR THE USE OF THIS SOFTWARE.  If software is
% modified to produce derivative works, such modified software should be
% clearly marked, so as not to confuse it with the version available from
% LANL.
%
% THIS SOFTWARE IS PROVIDED BY LOS ALAMOS NATIONAL SECURITY, LLC AND CONTRIBUTORS
% "AS IS" AND ANY EXPRESS OR IMPLIED WARRANTIES, INCLUDING, BUT NOT LIMITED TO,
% THE IMPLIED WARRANTIES OF MERCHANTABILITY AND FITNESS FOR A PARTICULAR PURPOSE
% ARE DISCLAIMED. IN NO EVENT SHALL LOS ALAMOS NATIONAL SECURITY, LLC OR
% CONTRIBUTORS BE LIABLE FOR ANY DIRECT, INDIRECT, INCIDENTAL, SPECIAL,
% EXEMPLARY, OR CONSEQUENTIAL DAMAGES (INCLUDING, BUT NOT LIMITED TO, PROCUREMENT
% OF SUBSTITUTE GOODS OR SERVICES; LOSS OF USE, DATA, OR PROFITS; OR BUSINESS
% INTERRUPTION) HOWEVER CAUSED AND ON ANY THEORY OF LIABILITY, WHETHER IN
% CONTRACT, STRICT LIABILITY, OR TORT (INCLUDING NEGLIGENCE OR OTHERWISE) ARISING
% IN ANY WAY OUT OF THE USE OF THIS SOFTWARE, EVEN IF ADVISED OF THE POSSIBILITY
% OF SUCH DAMAGE.



\subsubsection{\gufidirtrace}
\gufidirtrace generates trace files to allow for indexes to be easily
transfered to different locations rather than requiring entire trees
to be copied around.

\begin{table} [htb]
  \centering
  \begin{tabular}{l|l}
    Flag & Functionality \\
    \hline
    -h & help manual \\
    \hline
    -H & Show assigned input values \\
    \hline
    -v & version \\
    \hline
    -n \textless num\_threads\textgreater & define number of threads to use \\
    \hline
    -x & index/query xattrs \\
    \hline
    -d \textless delim\textgreater & delimiter (one char)  [use 'x' for 0x1E] \\
    \hline
    -k \textless filename\textgreater & file containing directory names to skip \\
    \hline
    -M \textless bytes\textgreater & target memory footprint \\
    \hline
    -s \textless path\textgreater & File name prefix for swap files \\
    \hline
    -C \textless count\textgreater & Number of subdirectories allowed to be \\
                                   & enqueued for parallel processing. Any \\
                                   & remainders will be processed in-situ \\
    \hline
    -e & compress work items \\
    \hline
    -q \textless basename\textgreater & Basename of file to keep track of during indexing \\
    \hline
  \end{tabular}
  \caption{\label{fig:Flags_for_dir2trace} \gufidirtrace Flags and Arguments}
\end{table}

\paragraph{Usage} ~\\\\
\gufidirtrace \texttt{[flags] input\_dir... output\_prefix} \\\\
Trace files with the name \texttt{output\_prefix.\$\{i\}}, where i $\in$ [0, number of threads), will be created.

% This file is part of GUFI, which is part of MarFS, which is released
% under the BSD license.
%
%
% Copyright (c) 2017, Los Alamos National Security (LANS), LLC
% All rights reserved.
%
% Redistribution and use in source and binary forms, with or without modification,
% are permitted provided that the following conditions are met:
%
% 1. Redistributions of source code must retain the above copyright notice, this
% list of conditions and the following disclaimer.
%
% 2. Redistributions in binary form must reproduce the above copyright notice,
% this list of conditions and the following disclaimer in the documentation and/or
% other materials provided with the distribution.
%
% 3. Neither the name of the copyright holder nor the names of its contributors
% may be used to endorse or promote products derived from this software without
% specific prior written permission.
%
% THIS SOFTWARE IS PROVIDED BY THE COPYRIGHT HOLDERS AND CONTRIBUTORS "AS IS" AND
% ANY EXPRESS OR IMPLIED WARRANTIES, INCLUDING, BUT NOT LIMITED TO, THE IMPLIED
% WARRANTIES OF MERCHANTABILITY AND FITNESS FOR A PARTICULAR PURPOSE ARE DISCLAIMED.
% IN NO EVENT SHALL THE COPYRIGHT HOLDER OR CONTRIBUTORS BE LIABLE FOR ANY DIRECT,
% INDIRECT, INCIDENTAL, SPECIAL, EXEMPLARY, OR CONSEQUENTIAL DAMAGES (INCLUDING,
% BUT NOT LIMITED TO, PROCUREMENT OF SUBSTITUTE GOODS OR SERVICES; LOSS OF USE,
% DATA, OR PROFITS; OR BUSINESS INTERRUPTION) HOWEVER CAUSED AND ON ANY THEORY OF
% LIABILITY, WHETHER IN CONTRACT, STRICT LIABILITY, OR TORT (INCLUDING NEGLIGENCE
% OR OTHERWISE) ARISING IN ANY WAY OUT OF THE USE OF THIS SOFTWARE, EVEN IF
% ADVISED OF THE POSSIBILITY OF SUCH DAMAGE.
%
%
% From Los Alamos National Security, LLC:
% LA-CC-15-039
%
% Copyright (c) 2017, Los Alamos National Security, LLC All rights reserved.
% Copyright 2017. Los Alamos National Security, LLC. This software was produced
% under U.S. Government contract DE-AC52-06NA25396 for Los Alamos National
% Laboratory (LANL), which is operated by Los Alamos National Security, LLC for
% the U.S. Department of Energy. The U.S. Government has rights to use,
% reproduce, and distribute this software.  NEITHER THE GOVERNMENT NOR LOS
% ALAMOS NATIONAL SECURITY, LLC MAKES ANY WARRANTY, EXPRESS OR IMPLIED, OR
% ASSUMES ANY LIABILITY FOR THE USE OF THIS SOFTWARE.  If software is
% modified to produce derivative works, such modified software should be
% clearly marked, so as not to confuse it with the version available from
% LANL.
%
% THIS SOFTWARE IS PROVIDED BY LOS ALAMOS NATIONAL SECURITY, LLC AND CONTRIBUTORS
% "AS IS" AND ANY EXPRESS OR IMPLIED WARRANTIES, INCLUDING, BUT NOT LIMITED TO,
% THE IMPLIED WARRANTIES OF MERCHANTABILITY AND FITNESS FOR A PARTICULAR PURPOSE
% ARE DISCLAIMED. IN NO EVENT SHALL LOS ALAMOS NATIONAL SECURITY, LLC OR
% CONTRIBUTORS BE LIABLE FOR ANY DIRECT, INDIRECT, INCIDENTAL, SPECIAL,
% EXEMPLARY, OR CONSEQUENTIAL DAMAGES (INCLUDING, BUT NOT LIMITED TO, PROCUREMENT
% OF SUBSTITUTE GOODS OR SERVICES; LOSS OF USE, DATA, OR PROFITS; OR BUSINESS
% INTERRUPTION) HOWEVER CAUSED AND ON ANY THEORY OF LIABILITY, WHETHER IN
% CONTRACT, STRICT LIABILITY, OR TORT (INCLUDING NEGLIGENCE OR OTHERWISE) ARISING
% IN ANY WAY OUT OF THE USE OF THIS SOFTWARE, EVEN IF ADVISED OF THE POSSIBILITY
% OF SUCH DAMAGE.



\subsubsection{\gufitraceindex}
\gufitraceindex is used to convert trace files into indexes. It is
essentially the same as \gufidirindex except it obtains data from
trace files instead of the filesystem being indexed.

Per thread trace files may be passed into \gufitraceindex
directly. Note that passing in too many trace files at once might
result in running out of file descriptors. Alternatively, the per
thread trace files may be concatenated in any order into a smaller
number of larger files for processing.

Extended attributes will be processed if they are found in the traces.
There is no need to tell \gufitraceindex to process them
with \texttt{-x}.

\begin{table} [htb]
  \centering
  \begin{tabular}{l|l}
    Flag & Functionality \\
    \hline
    -h & help manual \\
    \hline
    -H & Show assigned input values \\
    \hline
    -v & version \\
    \hline
    -n \textless num\_threads\textgreater & define number of threads to use \\
    \hline
    -d \textless delim\textgreater & delimiter (one char)  [use 'x' for 0x1E] \\
    \hline
    -M \textless bytes\textgreater & target memory footprint \\
    \hline
    -s \textless path\textgreater & File name prefix for swap files \\
    \hline
  \end{tabular}
  \caption{\label{fig:Flags_for_trace2index} \gufitraceindex Flags and Arguments}
\end{table}

\paragraph{Usage} ~\\\\
\gufitraceindex \texttt{[flags] trace\_file...} \texttt{/path/to/index} \\\\
Each source filesystem found in the trace files will be converted to an index placed underneath \indexroot.


\subsection{Distributed Indexing}
If indexing on a single node is not sufficient, it is possible to
distribute indexing across multiple nodes. The general process is:

\begin{enumerate}
\item Choose a level in the source tree to split up across nodes.
\item Get the sorted list of all directory basenames at the selected
  level.
\item Select continuous ranges of directories for distribution amongst
  the worker nodes (i.e. select directories names from \texttt{a}
  through \texttt{c} and \texttt{d} through \texttt{f}, not one
  directory starting with \texttt{a}, another directory starting with
  \texttt{z}, and another directory starting with \texttt{p}).
\item On each node, run \gufidirindex or \gufidirtrace with \texttt{-y
  level} and \texttt{-D start stop}. All other flags and arguments
  used should remain the same.
\item Run \gufidirindex or \gufidirtrace with \texttt{-z (level - 1)}
  to process the directories above the ones that were split across
  nodes.
\item Combine all of the output trees/traces together on the same
  filesystem for easy access.
\end{enumerate}

\gufidirindexdistributed and \gufidirtracedistributed have been
provided for convenience when running on clusters with Slurm. These
scripts have the additional step of waiting for subtree processing to
complete before scheduling the processing of the top level
directories.

\subsection{Index Placement}
The output (final positional) argument to \gufidirindex and
\gufitraceindex does not point to the directory where the index
starts. Rather, it points to the parent directory of where the index
starts\footnote{An empty db.db file is also placed here.}. This was
done in order to allow for users to pick the name of their GUFI search
space while leaving the index's top-most directory with the same name
as the source tree's top-most directory.

Placing multiple indexes under the same prefix is as simple as
maintaining the same output argument with different inputs
paths/traces.

Placing an index into the source tree is possible by pointing
\gufidirindex to the parent of the source tree.

Pointing the output of \gufidirindex to a location under the source
tree may result in the index creation never completing due to index
directories being created below the current tree walk location as the
source tree is being walked, causing the index directories to be
indexed, creating more index directories to be walked ad infinitum.

\input{sections/xattrs}
% This file is part of GUFI, which is part of MarFS, which is released
% under the BSD license.
%
%
% Copyright (c) 2017, Los Alamos National Security (LANS), LLC
% All rights reserved.
%
% Redistribution and use in source and binary forms, with or without modification,
% are permitted provided that the following conditions are met:
%
% 1. Redistributions of source code must retain the above copyright notice, this
% list of conditions and the following disclaimer.
%
% 2. Redistributions in binary form must reproduce the above copyright notice,
% this list of conditions and the following disclaimer in the documentation and/or
% other materials provided with the distribution.
%
% 3. Neither the name of the copyright holder nor the names of its contributors
% may be used to endorse or promote products derived from this software without
% specific prior written permission.
%
% THIS SOFTWARE IS PROVIDED BY THE COPYRIGHT HOLDERS AND CONTRIBUTORS "AS IS" AND
% ANY EXPRESS OR IMPLIED WARRANTIES, INCLUDING, BUT NOT LIMITED TO, THE IMPLIED
% WARRANTIES OF MERCHANTABILITY AND FITNESS FOR A PARTICULAR PURPOSE ARE DISCLAIMED.
% IN NO EVENT SHALL THE COPYRIGHT HOLDER OR CONTRIBUTORS BE LIABLE FOR ANY DIRECT,
% INDIRECT, INCIDENTAL, SPECIAL, EXEMPLARY, OR CONSEQUENTIAL DAMAGES (INCLUDING,
% BUT NOT LIMITED TO, PROCUREMENT OF SUBSTITUTE GOODS OR SERVICES; LOSS OF USE,
% DATA, OR PROFITS; OR BUSINESS INTERRUPTION) HOWEVER CAUSED AND ON ANY THEORY OF
% LIABILITY, WHETHER IN CONTRACT, STRICT LIABILITY, OR TORT (INCLUDING NEGLIGENCE
% OR OTHERWISE) ARISING IN ANY WAY OUT OF THE USE OF THIS SOFTWARE, EVEN IF
% ADVISED OF THE POSSIBILITY OF SUCH DAMAGE.
%
%
% From Los Alamos National Security, LLC:
% LA-CC-15-039
%
% Copyright (c) 2017, Los Alamos National Security, LLC All rights reserved.
% Copyright 2017. Los Alamos National Security, LLC. This software was produced
% under U.S. Government contract DE-AC52-06NA25396 for Los Alamos National
% Laboratory (LANL), which is operated by Los Alamos National Security, LLC for
% the U.S. Department of Energy. The U.S. Government has rights to use,
% reproduce, and distribute this software.  NEITHER THE GOVERNMENT NOR LOS
% ALAMOS NATIONAL SECURITY, LLC MAKES ANY WARRANTY, EXPRESS OR IMPLIED, OR
% ASSUMES ANY LIABILITY FOR THE USE OF THIS SOFTWARE.  If software is
% modified to produce derivative works, such modified software should be
% clearly marked, so as not to confuse it with the version available from
% LANL.
%
% THIS SOFTWARE IS PROVIDED BY LOS ALAMOS NATIONAL SECURITY, LLC AND CONTRIBUTORS
% "AS IS" AND ANY EXPRESS OR IMPLIED WARRANTIES, INCLUDING, BUT NOT LIMITED TO,
% THE IMPLIED WARRANTIES OF MERCHANTABILITY AND FITNESS FOR A PARTICULAR PURPOSE
% ARE DISCLAIMED. IN NO EVENT SHALL LOS ALAMOS NATIONAL SECURITY, LLC OR
% CONTRIBUTORS BE LIABLE FOR ANY DIRECT, INDIRECT, INCIDENTAL, SPECIAL,
% EXEMPLARY, OR CONSEQUENTIAL DAMAGES (INCLUDING, BUT NOT LIMITED TO, PROCUREMENT
% OF SUBSTITUTE GOODS OR SERVICES; LOSS OF USE, DATA, OR PROFITS; OR BUSINESS
% INTERRUPTION) HOWEVER CAUSED AND ON ANY THEORY OF LIABILITY, WHETHER IN
% CONTRACT, STRICT LIABILITY, OR TORT (INCLUDING NEGLIGENCE OR OTHERWISE) ARISING
% IN ANY WAY OUT OF THE USE OF THIS SOFTWARE, EVEN IF ADVISED OF THE POSSIBILITY
% OF SUCH DAMAGE.



\subsection{External Databases}
In addition to extended attributes, users may place \sqlite database
files into the source file system to be indexed and attached into GUFI
for querying. This effectively allows for users to associate and
query aribitrary data with filesystem metadata.

The extended attribute code uses the more generic external database
infrastructure to track the extended attributes that were not rolled
in.

\subsubsection{Usage}
All external database files for a set of data should have the same
basename. The files should be listed in a simple text file called
``external.gufi'' that is in the directory where a subset of the data
will be joined with GUFI data. If the path is relative, it will be
concatentated with the parent directory's path to produce an absolute
path.

\texttt{-q} may be passed in during indexing in order to verify that
an entry listed in ``external.gufi'' is indeed a database file.

One more more additional database files, referred to as the
``schema template", or ``template", containing a copy of the same
schema as each external database table should be maintained somewhere
for usage when querying. The template file should generally have empty
tables, but nothing prevents non-empty tables from being provided.

Note that ``external.gufi'' will also be included as an entry in the
index (this may change in the future).


\subsection{Location}
GUFI is expected to be used to query the indexes of many filesystems
all at once. However, the source filesystems are not expected to be
accessible from each other. In order for the indexes from many
disconnected systems to be queried at once, they should all be built
under a common directory on a single machine.

\subsubsection{Permissions}
If a GUFI tree is located on a different machine than the source
tree, the users and groups are probably not available on the machine
with the index on it. \texttt{/etc/passwd} should be populated with
the entries from the source machine and modified so that they do not
have a home directory, and are not allowed to log in
(\texttt{/sbin/nologin}). Similarly, \texttt{/etc/group} should be
updated with the source machine's groups and modified to remove any
unnecessary information.
