% This file is part of GUFI, which is part of MarFS, which is released
% under the BSD license.
%
%
% Copyright (c) 2017, Los Alamos National Security (LANS), LLC
% All rights reserved.
%
% Redistribution and use in source and binary forms, with or without modification,
% are permitted provided that the following conditions are met:
%
% 1. Redistributions of source code must retain the above copyright notice, this
% list of conditions and the following disclaimer.
%
% 2. Redistributions in binary form must reproduce the above copyright notice,
% this list of conditions and the following disclaimer in the documentation and/or
% other materials provided with the distribution.
%
% 3. Neither the name of the copyright holder nor the names of its contributors
% may be used to endorse or promote products derived from this software without
% specific prior written permission.
%
% THIS SOFTWARE IS PROVIDED BY THE COPYRIGHT HOLDERS AND CONTRIBUTORS "AS IS" AND
% ANY EXPRESS OR IMPLIED WARRANTIES, INCLUDING, BUT NOT LIMITED TO, THE IMPLIED
% WARRANTIES OF MERCHANTABILITY AND FITNESS FOR A PARTICULAR PURPOSE ARE DISCLAIMED.
% IN NO EVENT SHALL THE COPYRIGHT HOLDER OR CONTRIBUTORS BE LIABLE FOR ANY DIRECT,
% INDIRECT, INCIDENTAL, SPECIAL, EXEMPLARY, OR CONSEQUENTIAL DAMAGES (INCLUDING,
% BUT NOT LIMITED TO, PROCUREMENT OF SUBSTITUTE GOODS OR SERVICES; LOSS OF USE,
% DATA, OR PROFITS; OR BUSINESS INTERRUPTION) HOWEVER CAUSED AND ON ANY THEORY OF
% LIABILITY, WHETHER IN CONTRACT, STRICT LIABILITY, OR TORT (INCLUDING NEGLIGENCE
% OR OTHERWISE) ARISING IN ANY WAY OUT OF THE USE OF THIS SOFTWARE, EVEN IF
% ADVISED OF THE POSSIBILITY OF SUCH DAMAGE.
%
%
% From Los Alamos National Security, LLC:
% LA-CC-15-039
%
% Copyright (c) 2017, Los Alamos National Security, LLC All rights reserved.
% Copyright 2017. Los Alamos National Security, LLC. This software was produced
% under U.S. Government contract DE-AC52-06NA25396 for Los Alamos National
% Laboratory (LANL), which is operated by Los Alamos National Security, LLC for
% the U.S. Department of Energy. The U.S. Government has rights to use,
% reproduce, and distribute this software.  NEITHER THE GOVERNMENT NOR LOS
% ALAMOS NATIONAL SECURITY, LLC MAKES ANY WARRANTY, EXPRESS OR IMPLIED, OR
% ASSUMES ANY LIABILITY FOR THE USE OF THIS SOFTWARE.  If software is
% modified to produce derivative works, such modified software should be
% clearly marked, so as not to confuse it with the version available from
% LANL.
%
% THIS SOFTWARE IS PROVIDED BY LOS ALAMOS NATIONAL SECURITY, LLC AND CONTRIBUTORS
% "AS IS" AND ANY EXPRESS OR IMPLIED WARRANTIES, INCLUDING, BUT NOT LIMITED TO,
% THE IMPLIED WARRANTIES OF MERCHANTABILITY AND FITNESS FOR A PARTICULAR PURPOSE
% ARE DISCLAIMED. IN NO EVENT SHALL LOS ALAMOS NATIONAL SECURITY, LLC OR
% CONTRIBUTORS BE LIABLE FOR ANY DIRECT, INDIRECT, INCIDENTAL, SPECIAL,
% EXEMPLARY, OR CONSEQUENTIAL DAMAGES (INCLUDING, BUT NOT LIMITED TO, PROCUREMENT
% OF SUBSTITUTE GOODS OR SERVICES; LOSS OF USE, DATA, OR PROFITS; OR BUSINESS
% INTERRUPTION) HOWEVER CAUSED AND ON ANY THEORY OF LIABILITY, WHETHER IN
% CONTRACT, STRICT LIABILITY, OR TORT (INCLUDING NEGLIGENCE OR OTHERWISE) ARISING
% IN ANY WAY OUT OF THE USE OF THIS SOFTWARE, EVEN IF ADVISED OF THE POSSIBILITY
% OF SUCH DAMAGE.



\subsection{\gufiquery}

\gufiquery is the main tool used for accessing indexes.

\gufiquery processes each directory in parallel, passing user provided
SQL statements to the current directory's database. Because of this,
callers will need to know about GUFI's database and table schemas. We
expect only ``advanced'' users such as administrators and developers
to use \gufiquery directly. Users who do not know how GUFI functions
may call \gufiquery, but might not be able to use it properly.

\subsubsection{Flags}

\begin{longtable}{|l|p{0.5\linewidth}|}
  \hline
  Flag & Functionality \\
  \hline
  -h & help \\
  \hline
  -H & show assigned input values (debugging) \\
  \hline
  -T \textless SQL tsum\textgreater & SQL for tree-summary table \\
  \hline
  -S \textless SQL sum\textgreater & SQL for summary table \\
  \hline
  -E \textless SQL ent\textgreater & SQL for entries table \\
  \hline
  -a & AND/OR (SQL query combination) \\
  \hline
  -n \textless threads\textgreater & number of threads (default: 1) \\
  \hline
  -j & print the information in terse form \\
  \hline
  -o \textless out\_fname\textgreater & output file (one-per thread, with thread-id suffix) \\
  \hline
  -d \textless delim\textgreater & one char delimiter (default: \texttt{\textbackslash x1E}) \\
  \hline
  -O \textless out\_DB\textgreater & output DB \\
  \hline
  -u & prefix row with 1 int column count and each column with 1 octet type and 1 size\_t length \\
  \hline
  -I \textless SQL\_init\textgreater & SQL init \\
  \hline
  -F \textless SQL\_fin\textgreater & SQL cleanup \\
  \hline
  -y \textless min-level\textgreater & minimum level to descend to \\
  \hline
  -z \textless max-level\textgreater & maximum level to descend to \\
  \hline
  -J \textless SQL\_interm\textgreater & SQL for intermediate
    results (no default. ex.: \texttt{INSERT INTO \textless
    aggregate table name\textgreater \ SELECT * FROM \textless
    intermediate table name\textgreater}) \\
  \hline
  -K \textless create aggregate\textgreater & SQL to create the final aggregation table \\
  \hline
  -G \textless SQL\_aggregate\textgreater & SQL for aggregated
    results (no default. ex.: \texttt{SELECT * FROM
    \textless aggregate table name\textgreater}) \\
  \hline
  -m & Keep mtime and atime same on the database files \\
  \hline
  -B \textless buffer size\textgreater & size of each thread's output buffer in bytes \\
  \hline
  -w & open the database files in read-write mode instead of read only mode \\
  \hline
  -x & enable xattr processing \\
  \hline
  -k & file containing directory names to skip \\
  \hline
  -M \textless bytes\textgreater & target memory footprint \\
  \hline
  -s \textless path\textgreater & File name prefix for swap files \\
  \hline
  -p \textless path\textgreater & Source path prefix for \%s in SQL \\
  \hline
  -e & compress work items \\
  \hline
  -Q \textless basename\textgreater & External database file basename, \\
  \ \ \ \textless table\textgreater & per-attach table name, \\
  \ \ \ \textless template\textgreater.\textless table\textgreater & template + table name, \\
  \ \ \ \textless view\textgreater & and the resultant view \\
  \hline
  \caption{\label{tab:widgets} \gufiquery Flags and Arguments}\\
\end{longtable}

\subsubsection{Flag/Table Associations}
A \gufiquery call should use at least one of the \texttt{-T},
\texttt{-S}, and \texttt{-E} flags to pass in SQL
statements\footnote{Not passing in any SQL statements results in
  \gufiquery simply walking the tree and filling up the inode and
  dentries caches.}.

\begin{itemize}
\item \texttt{-T} should be used to query the \treesummary table.
\item \texttt{-S} should be used to query the \summary table and its
  variants.
\item \texttt{-E} should be used to query the \entries table and its
  variants.
\end{itemize}

Note that user provided SQL statements are passed directly into
\sqlite\footnote{Anything that can be done with SQL can also be done on
  the databases in an index. To prevent accidental modifications,
  databases default to opening in read-only mode.} and thus
associations between flags and tables are not
enforced\footnote{Querying tables/views with the wrong flags may
  result in unexpected output.}.

Prior to rolling up, the \pentries view can be treated as an optional
improvement to the \entries table. After rolling up, the \pentries
view will have been updated extensively and will contain both the
original \entries data as well as the rolled up data. Querying the
\entries table of a rolled up index will return a subset of the total
results that exist in the index since it only contains the current
directory's information and subdirectories are not traversed. The
\summary table was updated directly, so querying it will return all
directory information.

When querying, the caller should choose tables and views based on
whether or not full paths will be queried. If no, the \summary table
and \pentries, \xpentries, or \epentries views should be used. If yes,
the \vrpentries and \vrsummary views and their variants should be
used. The \vrpentries and \vrsummary views contain data from the
\summary table that allow for the path of each row relative to the
starting path to be generated using the \rpath SQL function, whether
or not the index has been rolled up, with consistent input arguments
(see Table~\ref{tab:sqlwcontext} for details). These views and
function were set up to help simpliy the queries that users provide.

\subsubsection{Short Circuiting with Compound Queries}
At each directory, the set of user provided SQL statements run on the
db.db file in the following order: if \texttt{-T} was provided, it
will run if the optional \treesummary table exists in the database. If
\texttt{-T} was not run or returned at least one row of results,
\texttt{-S} will run if it was provided. If \texttt{-S} was not
provided or returned at least one row of results, \texttt{-E} will run
if it was provided.

Conversely, if \texttt{-T} runs and does not return any results,
processing is stopped before running the \texttt{-S} query. If
\texttt{-S} does not return any results, the \texttt{-E} query is not
run. This allows for threads to short circuit the processing of a
directory if it is determined early on that no rows would be obtained
from a later query. For example, if a query is searching for files
larger than 1MB, but the \texttt{-S} query found that the range of
file sizes is 1KB to 5KB, there is no need to run the \texttt{-E}
query to get rows, since no rows will match in any case.

To turn off short circuiting and always run all queries for each
directory, pass the \texttt{-a} flag to \gufiquery.

\subsubsection{User String Replacement}
\label{sec:user_strings}
Queries may have string replacements done to them before being run on
a particular directory.

After calling \texttt{setstr(`key', value)} in one SQL input argument,
\texttt{\{key\}} can be used in a later SQL input argument to do
string replacements. The key can be any string with at least 1
character. Duplicate keys will replace previously defined values.

Note that the key and value will only be available to the thread where
it was processed. Each thread should set their own instances of the
key.
\\\\
\noindent Example Usage:
\begin{table}[H]
  \centering
  \begin{tabular}{ll}
    \gufiquery & \texttt{-S "SELECT setstr(`key', (SELECT 123));"} \\
               & \texttt{-E "SELECT \{key\};"} \\
               & \indexroot \\
  \end{tabular}
\end{table}

SQL statements combining setting and using values into one input
argument will not work because the entire string will have been parsed
by \sqlite before replacements can happen. \\

\noindent Bad Usage:
\begin{table}[H]
  \centering
  \begin{tabular}{ll}
    \gufiquery & \texttt{-S "SELECT setstr(`key', (SELECT 123)); SELECT \{key\};"} \\
               & \indexroot \\
  \end{tabular}
\end{table}

Additionally, a few user strings are predefined for convenience. They
were previously part of the deprecated \texttt{\% Formatting}, and
have been moved here.

\begin{table}[H]
  \centering
  \begin{tabular}{|l|l|}
    \hline
    User String & Explanation \\
    \hline
    \texttt{n} & Replace with the current index directory's name \\
    \hline
    \texttt{i} & Replace with the current index directory's path \\
    \hline
    \texttt{s} & Replace with the source path of this directory \\
               & (Requires \texttt{-p}) \\
    \hline
  \end{tabular}
  \caption{Predefined User Strings}
\end{table}

These strings may be redefined by the user. However, \texttt{n} and
\texttt{i} will be reset at the beginning of the next directory's
processing.

\subsubsection{Extended Attributes}
\label{sec:query_xattrs}
When querying for xattrs, pass \texttt{-x} to \gufiquery to build the
\xattrs view for querying. This view is a SQL \texttt{UNION} of rolled
in xattrs and any external xattr databases that successfully
attaches. Attaching the database files checks the permissions of the
xattrs. \texttt{UNION} removes duplicate entries that showed up in
multiple external databases, leaving a unique set of xattrs accessible
by the caller.

The \xentries, \xpentries, \xsummary, \vrxpentries, and \vrxsummary
views are convenience views generated so that users do not have to
perform joins themselves. They are \entries, \pentries, \summary,
\vrpentries, and \vrsummary enhanced with the \texttt{xattr\_name} and
\texttt{xattr\_value} columns.

Note that entries with multiple extended attributes will return
multiple times in this view.

\subsubsection{External Databases}
Attaching external data to GUFI allows for arbitrary data to be
associated with filesystem data. In order to remain flexible and not
prescribe schema requirements, using this feature is more complex than
using other GUFI features.

Each external database file is presented as a view concatentating it
to a table in the template database file. When a directory does not
have a database file specified for attaching to a given view, the view
is comprised of only the table from the template file.

Pass \texttt{-I} and \texttt{-Q} to \gufiquery. \texttt{-Q} has 4
positional arguments:

\begin{table}[h!]
  \centering
  \begin{tabular}{|l|l|}
    \hline
    \texttt{-Q} Argument & Explanation \\
    \hline
    \texttt{basename} & The basename of the database file in the current directory \\
                      & to attach \\
    \hline
    \texttt{table} & The table in the database file to attach \\
    \hline
    \texttt{template.table} & The template file's attach name (specified in \texttt{-I}) and table \\
                            & name with a matching schema \\
    \hline
    \texttt{view} & The name of the view that will be made available for \\
                  & \texttt{-S} and \texttt{-E} \\
    \hline
  \end{tabular}
\end{table}

Finally, in \texttt{-S} and \texttt{-E}, the caller
should join the view named in \texttt{-Q} with new GUFI views
\esummary, \epentries, \exsummary, \expentries, \evrsummary,
\evrxsummary, \evrpentries, and \evrxpentries to pull data. The
recommended columns to join on are \texttt{name} and \texttt{type} to
reduce the likelihood of joining to the wrong rows.
\\\\
Example Usage:
\begin{table}[H]
  \centering
  \begin{tabular}{ll}
    \gufiquery & \texttt{-I "ATTACH template.db AS template;"} \\
               & \texttt{-Q "external.db" "table" "template.table" "ext"} \\
               & \texttt{-E "SELECT * FROM evrpentries} \\
               & \texttt{\ \ \ \ LEFT JOIN ext ON (evrpentries.name == ext.name)} \\
               & \texttt{\ \ \ \ AND (evrpentries.type == ext.type);"} \\
               & \indexroot \\
  \end{tabular}
\end{table}

For each \texttt{view} that is to be made available for querying,
\texttt{-I} should be updated with more \texttt{ATTACH} SQL
statements, another \texttt{-Q} should be passed in, and each query
flag should be updated to use the new views.
\\\\
Multiple External Databases Example:
\begin{table}[H]
  \centering
  \begin{tabular}{ll}
    \gufiquery & \texttt{-I "ATTACH template1.db AS template1;} \\
               & \texttt{\ \ \ \ ATTACH template2.db AS template2;"} \\
               & \texttt{-Q "externall.db" "table" "template1.table" "ext1"} \\
               & \texttt{-Q "external2.db" "table" "template2.table" "ext2"} \\
               & \texttt{-E "SELECT * FROM evrpentries} \\
               & \texttt{\ \ \ \ LEFT JOIN ext1 ON (evrpentries.name == ext1.name)} \\
               & \texttt{\ \ \ \ AND (evrpentries.type == ext1.type)} \\
               & \texttt{\ \ \ \ LEFT JOIN ext2 ON (evrpentries.name == ext2.name)} \\
               & \texttt{\ \ \ \ AND (evrpentries.type == ext2.type);"} \\
               & \indexroot \\
  \end{tabular}
\end{table}

\subsubsection{Aggregation}
There are cases where independent per-thread results are not
desirable, such as when sorting or summing, where the results from
querying the index must be aggregated for final processing.

In order to handle these situations, the \texttt{-I} flag should be
used to create per-thread intermediate tables that are written to by
\texttt{-T}, \texttt{-S}, and \texttt{-E}. The intermediate table
results will then be aggregated using \texttt{-J} into the final table
created by \texttt{-K}. The rows stored in the final table are
processed one last time as a whole, rather than as results from
independent threads, using \texttt{-G}.

\subsubsection{Per-Thread Output Files}
The \texttt{-o} flag causes results to be outputted to text
files. When outputting in parallel, per-thread output files are
created with the thread id appended to the filename provided with the
flag. When aggregating, the aggregate results are written to the
filename specified by \texttt{-o} with no filename modifications.

\subsubsection{Per-Thread Output Database Files}
The \texttt{-O} flag allows for results to be written to SQLite
database files instead of text files. The resulting filenames follow
the same creation rules as \texttt{-o}. However, the queries passed
into \gufiquery need modifications. When writing in parallel,
\texttt{-I} is needed to create the table for each per-thread output
database. \texttt{-T}, \texttt{-S}, and \texttt{-E} should be modified
to write to the per-thread tables in the same way as writing to the
intermediate tables when aggregating. When writing aggregate results
to a database, \texttt{-G} is not needed as \texttt{-J} already wrote
the results into the aggregate table. However, \texttt{-G} may still
be provided to get results during the \gufiquery in addition to
queries on the results database file later on.

Output database files may be passed to the \querydbs executable for
further processing.

\subsubsection{Example Calls}

\begin{table}[H]
  \centering
  \caption{Parallel Results}
  \begin{tabular}{|l|l|}
    \hline
    Output     & \gufiquery \indexroot \\
    \hline
    \stdout    & \texttt{-S "SELECT * FROM \vrsummary;"} \\
    \hline
    Per-thread & \texttt{-E "SELECT * FROM \vrpentries;"} \\
    Files      & \texttt{-o results} \\
    \hline
    Per-thread & \texttt{-I "CREATE  TABLE results(name TEXT, size INT64);"} \\
    Database   & \texttt{-S "INSERT INTO results SELECT name, size FROM \vrsummary;"} \\
    Files      & \texttt{-E "INSERT INTO results SELECT name, size FROM \vrpentries;"} \\
               & \texttt{-O results} \\
    \hline
  \end{tabular}
\end{table}

\begin{table}[H]
  \centering
  \caption{Aggregate Results}
  \begin{tabular}{|l|l|}
    \hline
    Output   & \gufiquery \indexroot \\
    \hline
    \stdout  & \texttt{-I "CREATE TABLE intermediate(name TEXT, size INT64);"} \\
             & \texttt{-S "INSERT INTO intermediate SELECT name, size FROM \vrsummary;"} \\
             & \texttt{-K "CREATE TABLE aggregate(name TEXT, size INT64);"} \\
             & \texttt{-J "INSERT INSERT aggregate SELECT * FROM intermediate;"} \\
             & \texttt{-G "SELECT * FROM aggregate ORDER BY size DESC;"} \\
    \hline
    Single   & \texttt{-I "CREATE TABLE intermediate(name TEXT, size INT64);"} \\
    File     & \texttt{-E "INSERT INTO intermediate SELECT name, size FROM \vrpentries;"} \\
             & \texttt{-K "CREATE TABLE aggregate(name TEXT, size INT64);"} \\
             & \texttt{-J "INSERT INSERT aggregate SELECT * FROM intermediate;"} \\
             & \texttt{-G "SELECT * FROM aggregate ORDER BY size DESC;"} \\
             & \texttt{-o results} \\
    \hline
    Single   & \texttt{-I "CREATE TABLE intermediate(name TEXT, size INT64);"} \\
    Database & \texttt{-S "INSERT INTO intermediate SELECT name, size FROM \vrsummary;"} \\
    File     & \texttt{-E "INSERT INTO intermediate SELECT name, size FROM \vrpentries;"} \\
             & \texttt{-K "CREATE TABLE aggregate(name TEXT, size INT64);"} \\
             & \texttt{-J "INSERT INSERT aggregate SELECT * FROM intermediate;"} \\
             & \texttt{-O results} \\
    \hline
  \end{tabular}
\end{table}

\subsubsection{Behavior When Traversing Rolled Up Indexes}
When \gufiquery detects that a directory being processed has been
rolled up, the thread processing that directory does not enqueue work
to descend further down into the tree, as all of the data underneath
the current directory is available in the current directory. While
this may seem to only reduce tree traversal by ``some" amount, in
practise, rolling up indexes of real filesystems reduces the number of
directories (and thus the bottlenecks listed in
Section~\ref{subsec:bottlenecks}) that need to be processed by over
99\%, significantly reducing index query time.

Queries are not expected to have to change too often when switching
between unmodified indexes and rolled up indexes. If they do, they
should not have to change by much.

\subsubsection{Visualizing the Workflow}
\begin{figure}[H]
  \centering
  \includegraphics[width=\textwidth]{images/gufi_query_main.png}
  \caption{Workflow of \gufiquery}
\end{figure}

\paragraph{\processdir}
The core of \gufiquery is the \processdir function. This is where the
-T, -S, and -E flags are processed. Multiple instances of this
function are run in parallel via the thread pool in order to quickly
traverse and process an index.

\begin{figure}[H]
  \centering
  \includegraphics[width=\textwidth]{images/gufi_query_processdir.png}
  \caption{Workflow of \processdir}
\end{figure}

\paragraph{\querydb}
The \querydb macro is used to execute SQL statements and handle errors.

\begin{figure}[H]
  \centering
  \includegraphics[width=\textwidth]{images/querydb_macro.png}
  \caption{\querydb macro workflow}
\end{figure}

\subsubsection{Distributed Querying}
If querying from a single node is not sufficient, it is possible to
distribute querying across multiple nodes. Similar to distributed
indexing, the general process is:

\begin{enumerate}
\item Choose a level in the index to split up across nodes.
\item Write the directory paths (without the starting path) at the
  selected level to a file using \texttt{find -mindepth <level>
    -maxdepth <level> -type d -printf "\%P\textbackslash n"}.
\item Select ranges of directories from the previous step and write
  the ranges to separate files. These files represent the work a
  single node will process.
\item On each node, run \gufiquery with \texttt{-y level}, \texttt{-D
  <filename.of.range>}, and (\texttt{-o output}), making sure that output
  from different nodes do not write to the same files. All other flags
  and arguments used should remain the same.
\item Run \gufiquery with \texttt{-z (level - 1)} to process the
  directories above the ones that were split across nodes. Also write
  the output to text files (\texttt{-o}).
\item Combine all of the output files containing the partial results
  to get the complete results.
\end{enumerate}

\gufiquerydistributed has been provided for convenience when running
on clusters with Slurm or SSH. As with the distributed indexing
scripts, this script has the additional step of waiting for subtree
processing to complete before scheduling the processing of the top
level directories.
