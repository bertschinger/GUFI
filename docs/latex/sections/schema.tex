% This file is part of GUFI, which is part of MarFS, which is released
% under the BSD license.
%
%
% Copyright (c) 2017, Los Alamos National Security (LANS), LLC
% All rights reserved.
%
% Redistribution and use in source and binary forms, with or without modification,
% are permitted provided that the following conditions are met:
%
% 1. Redistributions of source code must retain the above copyright notice, this
% list of conditions and the following disclaimer.
%
% 2. Redistributions in binary form must reproduce the above copyright notice,
% this list of conditions and the following disclaimer in the documentation and/or
% other materials provided with the distribution.
%
% 3. Neither the name of the copyright holder nor the names of its contributors
% may be used to endorse or promote products derived from this software without
% specific prior written permission.
%
% THIS SOFTWARE IS PROVIDED BY THE COPYRIGHT HOLDERS AND CONTRIBUTORS "AS IS" AND
% ANY EXPRESS OR IMPLIED WARRANTIES, INCLUDING, BUT NOT LIMITED TO, THE IMPLIED
% WARRANTIES OF MERCHANTABILITY AND FITNESS FOR A PARTICULAR PURPOSE ARE DISCLAIMED.
% IN NO EVENT SHALL THE COPYRIGHT HOLDER OR CONTRIBUTORS BE LIABLE FOR ANY DIRECT,
% INDIRECT, INCIDENTAL, SPECIAL, EXEMPLARY, OR CONSEQUENTIAL DAMAGES (INCLUDING,
% BUT NOT LIMITED TO, PROCUREMENT OF SUBSTITUTE GOODS OR SERVICES; LOSS OF USE,
% DATA, OR PROFITS; OR BUSINESS INTERRUPTION) HOWEVER CAUSED AND ON ANY THEORY OF
% LIABILITY, WHETHER IN CONTRACT, STRICT LIABILITY, OR TORT (INCLUDING NEGLIGENCE
% OR OTHERWISE) ARISING IN ANY WAY OUT OF THE USE OF THIS SOFTWARE, EVEN IF
% ADVISED OF THE POSSIBILITY OF SUCH DAMAGE.
%
%
% From Los Alamos National Security, LLC:
% LA-CC-15-039
%
% Copyright (c) 2017, Los Alamos National Security, LLC All rights reserved.
% Copyright 2017. Los Alamos National Security, LLC. This software was produced
% under U.S. Government contract DE-AC52-06NA25396 for Los Alamos National
% Laboratory (LANL), which is operated by Los Alamos National Security, LLC for
% the U.S. Department of Energy. The U.S. Government has rights to use,
% reproduce, and distribute this software.  NEITHER THE GOVERNMENT NOR LOS
% ALAMOS NATIONAL SECURITY, LLC MAKES ANY WARRANTY, EXPRESS OR IMPLIED, OR
% ASSUMES ANY LIABILITY FOR THE USE OF THIS SOFTWARE.  If software is
% modified to produce derivative works, such modified software should be
% clearly marked, so as not to confuse it with the version available from
% LANL.
%
% THIS SOFTWARE IS PROVIDED BY LOS ALAMOS NATIONAL SECURITY, LLC AND CONTRIBUTORS
% "AS IS" AND ANY EXPRESS OR IMPLIED WARRANTIES, INCLUDING, BUT NOT LIMITED TO,
% THE IMPLIED WARRANTIES OF MERCHANTABILITY AND FITNESS FOR A PARTICULAR PURPOSE
% ARE DISCLAIMED. IN NO EVENT SHALL LOS ALAMOS NATIONAL SECURITY, LLC OR
% CONTRIBUTORS BE LIABLE FOR ANY DIRECT, INDIRECT, INCIDENTAL, SPECIAL,
% EXEMPLARY, OR CONSEQUENTIAL DAMAGES (INCLUDING, BUT NOT LIMITED TO, PROCUREMENT
% OF SUBSTITUTE GOODS OR SERVICES; LOSS OF USE, DATA, OR PROFITS; OR BUSINESS
% INTERRUPTION) HOWEVER CAUSED AND ON ANY THEORY OF LIABILITY, WHETHER IN
% CONTRACT, STRICT LIABILITY, OR TORT (INCLUDING NEGLIGENCE OR OTHERWISE) ARISING
% IN ANY WAY OUT OF THE USE OF THIS SOFTWARE, EVEN IF ADVISED OF THE POSSIBILITY
% OF SUCH DAMAGE.


\section{Database Schema}
\label{sec:schema}
GUFI stores extracted metadata in SQLite3 database files. Each
directory contains a database file named db.db that contains a set of
tables and views designed to facilitate efficient querying of
metadata.

\subsection{\entries}
The \entries table contains file and link metadata extracted from the
source filesystem using \lstat, \readlink, and \listxattr. There are
also a few OS specific columns (\texttt{oss*}) that are unused in base
GUFI. Note that the inode column is a string and not an integer as one
might expect. As a general rule, do not query this table directly.

\subsection{Directory \summary}
The directory \summary table contains the metadata of the current
directory. Additionally, it contains columns that summarizes the
\entries table located in the same database file, such as minimum and
maximum file sizes, uids, and gids. These summary columns can be used
to determine whether or not the \entries table needs to be queried at
all. Note that the inode and pinode colums are strings and not
integers as one might expect.

\begin{figure} [h!]
\centering
\includegraphics[width=1.0\textwidth]{images/Database_Schemas.png}
\caption{\label{fig:Database Schema}Database schema}
\end{figure}

\subsection{\pentries}
The original \pentries view was defined as:
\\\\
\texttt{SELECT entries.*, summary.inode AS pinode, summary.pinode AS
  ppinode FROM entries, summary};
\\\\
This was done in order to not store the parent inode of each entry,
which would be the same for every entry.

Users should prefer query the \pentries view over the \entries table
in order to obtain complete sets of data to query on, and should use
\pentries whenever paths are not required.

\subsubsection{\pentriesrollup}
In order to simplify rolling up indexes (see
Section~\ref{sec:rollup}), the \pentries view was modified to also
union with the \pentriesrollup table, which contains all child
\pentries views copied into the current directory's database file.

\subsection{\vrsummary}
The \vrsummary view is the \summary table with a handful of columns
repeated as aliases. This was done so that the \texttt{rpath} SQL
function can be called to generate the path of a given directory with
the same view and query whether or not the index has been rolled up,
and thus should be preferred over querying the \summary table. Using
the \summary table directly with a rolled up index is possible, but
will complicate queries.

\subsection{\texttt{vrentries}}
The \texttt{vrentries} view does not exist because it would be a
misuse of the schema: the entries table itself is never rolled up, so
querying a \texttt{vr} version of it would result in incorrect output.

\subsection{\vrpentries}
The \vrpentries view is the \pentries view with a few \summary table
columns aliased. This allows for the \texttt{rpath} SQL function can
be called to generate the path of a given directory with the same view
and query whether or not the index has been rolled up, and thus should
be preferred over querying \entries and \pentries. \texttt{rpath} with
the \vrpentries view is called with the same arguments that are used
with \vrsummary. Using the \pentries view directly with a rolled up
index is possible, but will complicate queries.

\subsection{\treesummary}
Similar to the directory \summary table, GUFI also provides
functionality to generate \treesummary tables. Instead of summarizing
only the data found in the entries table of the current directory, the
\treesummary table summarizes the contents of the entire subtree,
allowing for queries to completely skip processing entire
subtrees. See Section~\ref{sec:treesummary} for details.

\subsection{\vsummarydir}
The \vsummarydir view provides access to the entire directory summary
and not a partial directory summary (say by user or group).

\subsection{\vsummaryuser}
The \vsummaryuser view provides access to the directory summary for
each user (if this summary has been populated (not by default but
easily populatable via a query)).

\subsection{\vsummarygroup}
The \vsummarygroup view provides access to the directory summary for
each group (if this summary row has been created (not by default but
easily created via a query)).

\subsection{\vtsummarydir}
The vtsummarydir view provides access to the entire tree directory
summary and not a partial directory summary (say by user or group).

\subsection{\vtsummaryuser}
The \vtsummaryuser view provides access to the tree directory summary
for each user (if this summary row has been created (not by default
but easily created via a query)).

\subsection{\vtsummarygroup}
The \vtsummarygroup view provides access to the tree directory summary
for each group (if this summary has been populated (not by default but
easily populatable via a query)).

\subsection{\xattrs}
With the addition of extended attributes support, several more tables
and views were added into db.db. From the user's perspective, the
\xattrs view has been created and joined with \entries, \pentries,
\summary, \vrpentries, and \vrsummary to create the convenience views
\xentries, \xpentries, \xsummary, \vrxpentries, and \vrxsummary. These
views are always available whether or not extended attribute
processing has been enabled. See Sections~\ref{sec:xattr_schema} and
\ref{sec:query_xattrs} for details. Using \vrxpentries and \vrxsummary
should be preferred over all previously mentioned views when paths are
required.

\subsection{External Databases}
With the addition of external (user) database support, several more
views were added into db.db. External database \texttt{e} variants of
\summary, \pentries, \xsummary, \xpentries, \vrsummary, \vrxsummary,
\vrpentries, and \vrxpentries were added: \esummary, \epentries,
\exsummary, \expentries \evrsummary, \evrxsummary, \evrpentries,
\evrxpentries. These views should be joined with the external
databases that were attached. Using these views should be preferred
over all previously mentioned views when paths are required.
